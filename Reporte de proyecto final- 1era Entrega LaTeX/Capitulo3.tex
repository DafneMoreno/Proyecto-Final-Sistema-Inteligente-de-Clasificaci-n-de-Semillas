\section{Introducción}
La clasificación de semillas ha evolucionado significativamente a lo largo del tiempo. En sus inicios, los agricultores dependían exclusivamente de la inspección visual y manual, un proceso lento y propenso a errores. A mediados del siglo XX, se introdujeron métodos mecánicos como tamices y separadores por peso, que mejoraron la eficiencia pero aún carecían de precisión para características sutiles. En la década de 1980, surgieron las primeras aplicaciones de visión por computadora en la agricultura, marcando el inicio de una nueva era. Sin embargo, estos sistemas eran costosos y complejos, limitando su adopción generalizada. 
En el último lustro, la visión por ordenador ha sido ampliamente utilizada en el área de la ingeniería agromótica para la inspección, seguimiento, riego y cosecha de los productos agrícolas. Actualmente se están cambiando las operaciones manuales tradicionales como son la mano de obra lenta y propensa a errores de los seres humanos \cite{hernandez2016comparacion}. La clasificación precisa de semillas tiene un impacto económico sustancial en la industria agrícola. Según un estudio realizado por la FAO en 2020, las pérdidas debidas a la clasificación incorrecta de semillas pueden alcanzar hasta el 10\% de la producción total en algunos cultivos \cite{mensah2021adoption}. Esto se traduce en millones de dólares en pérdidas anuales para los agricultores y la industria alimentaria. Por otro lado, la implementación de métodos de clasificación más precisos ha demostrado aumentar la productividad entre un 15\% y un 25\%, dependiendo del cultivo. Por ejemplo, en un estudio de caso realizado en campos de trigo en Estados Unidos, la adopción de tecnologías avanzadas de clasificación de semillas resultó en un aumento del rendimiento del 18\% y una reducción de costos del 12\% en un período de tres años. 



La clasificación precisa de semillas es fundamental para garantizar la calidad y el rendimiento de los cultivos. Este proyecto se enfoca en desarrollar un sistema de clasificación de semillas utilizando una cámara ESP32-CAM, que combina tecnología de visión por computadora con un dispositivo compacto y económico. La motivación principal de esta investigación es mejorar la eficiencia y precisión en la selección de semillas, lo que puede tener un impacto significativo en la productividad agrícola. Uno de los principales retos que se enfrentarán es la gran variabilidad en las características físicas de las semillas, incluyendo tamaño, color y forma. Estas variaciones pueden ser sutiles entre diferentes variedades de la misma especie, o incluso entre semillas de la misma variedad debido a factores ambientales durante su desarrollo. Además, las condiciones de iluminación juegan un papel crucial en la captura de imágenes precisas. La iluminación inconsistente o inadecuada puede alterar la percepción del color y la textura de las semillas, lo que podría llevar a clasificaciones erróneas. Otro desafío importante es la necesidad de desarrollar algoritmos robustos capaces de manejar estas variaciones y adaptarse a diferentes tipos de semillas sin requerir una recalibración constante. Estos algoritmos deben ser lo suficientemente flexibles para distinguir entre características críticas y variaciones normales, y al mismo tiempo ser eficientes para procesar grandes volúmenes de semillas en tiempo real. La integración de estas capacidades en un dispositivo compacto y económico como la ESP32-CAM añade una capa adicional de complejidad, requiriendo una optimización cuidadosa de los recursos computacionales y de memoria.

Este proyecto se alinea estrechamente con los objetivos de agricultura sostenible y seguridad alimentaria establecidos por las Naciones Unidas en sus Objetivos de Desarrollo Sostenible (ODS). La clasificación precisa de semillas contribuye directamente al ODS 2 (Hambre Cero) y al ODS 12 (Producción y Consumo Responsables). Al mejorar la eficiencia en la selección de semillas, se reduce el desperdicio de recursos como agua, tierra y fertilizantes, lo que a su vez disminuye la huella ecológica de la producción agrícola. Además, la mejora en la calidad de las semillas seleccionadas puede llevar a cultivos más resistentes a enfermedades y condiciones climáticas adversas, contribuyendo así a la resiliencia alimentaria frente al cambio climático. La FAO estima que la adopción de tecnologías como la propuesta en este proyecto podría contribuir a reducir el hambre en un 20\% en regiones en desarrollo para el año 2030, al tiempo que disminuye el uso de agua en la agricultura en un 15\% \cite{trendov2019tecnologias}.

En el campo de la visión por computadora y el aprendizaje automático aplicado a la agricultura, existen diversos algoritmos y librerías que pueden ser fundamentales para el desarrollo de un sistema de clasificación de semillas utilizando una ESP32-CAM. Estos componentes software son esenciales para procesar las imágenes capturadas y realizar una clasificación precisa y eficiente. Entre los algoritmos más relevantes para este tipo de proyecto se encuentran los de procesamiento de imágenes y los de aprendizaje automático. En cuanto a los algoritmos de clasificación, las Máquinas de Soporte Vectorial (SVM) y los árboles de decisión son opciones populares debido a su eficacia y relativa ligereza computacional. Para implementar estos algoritmos, varias librerías de código abierto son invaluables. OpenCV, una de las más utilizadas en visión por computadora, ofrece una amplia gama de funciones para el procesamiento de imágenes y la implementación de algoritmos de visión. Su versión para sistemas embebidos, OpenCV-lite, es especialmente relevante para dispositivos con recursos limitados como la ESP32-CAM \cite{garcia2015learning}.
TensorFlow Lite es otra librería crucial, diseñada específicamente para ejecutar modelos de aprendizaje automático en dispositivos con restricciones de recursos. Esta librería permite la implementación eficiente de redes neuronales y otros algoritmos de aprendizaje automático en la ESP32-CAM.
Para el desarrollo del software en la ESP32-CAM, el framework ESP-IDF (Espressif IoT Development Framework) proporciona un conjunto de herramientas y librerías optimizadas para este hardware específico. Este framework facilita la integración de los algoritmos de visión por computadora con las capacidades de la cámara y el procesamiento de la ESP32-CAM. Además, librerías como NumPy y SciPy, aunque no se ejecutan directamente en la ESP32-CAM, son invaluables para el desarrollo y entrenamiento de modelos en un entorno de desktop antes de su implementación en el dispositivo embebido. Estas librerías ofrecen funciones avanzadas de cálculo numérico y procesamiento de señales que son fundamentales en el desarrollo de algoritmos de clasificación.

La selección y adaptación adecuada de estos algoritmos y librerías es crucial para desarrollar un sistema de clasificación de semillas eficiente y preciso utilizando la ESP32-CAM. El desafío radica en optimizar estos componentes software para que funcionen de manera efectiva dentro de las limitaciones de memoria y potencia de procesamiento del dispositivo, manteniendo al mismo tiempo un alto nivel de precisión en la clasificación.

El objetivo principal de este estudio es diseñar y implementar un sistema automatizado capaz de clasificar diferentes tipos de semillas con alta precisión y contabilizarlas. Además, se busca crear una solución accesible y fácil de usar para agricultores y productores de semillas, que pueda integrarse fácilmente en sus procesos actuales. Este enfoque tiene el potencial de reducir el tiempo y los recursos necesarios para la clasificación manual de semillas, al tiempo que minimiza los errores humanos.
El problema que se pretende resolver es la dificultad y el costo asociados con la clasificación manual de semillas, que a menudo resulta en inexactitudes y requiere mucho tiempo. Los métodos tradicionales de clasificación pueden ser subjetivos y propensos a errores, lo que afecta la calidad de los cultivos y, en última instancia, la producción agrícola. Al automatizar este proceso mediante el uso de tecnología de visión por computadora, se espera mejorar significativamente la precisión y la eficiencia de la clasificación de semillas.


Este proyecto se basa en investigaciones previas en el campo de la visión por computadora aplicada a la agricultura. Por ejemplo, el trabajo de García-Mateos \cite{garcia2015study} demostró la eficacia de los sistemas de visión por computadora para la clasificación de frutas y verduras. Además, estudios como el de Mahajan \cite{mahajan2015image} han explorado el uso de técnicas de aprendizaje automático para la clasificación de semillas, sentando las bases para nuestra investigación.
Al combinar estos avances con la versatilidad y accesibilidad de la ESP32-CAM, este proyecto busca crear una solución innovadora que pueda tener un impacto real en la industria agrícola. Se espera que los resultados de esta investigación no solo mejoren la eficiencia en la clasificación de semillas, sino que también abran nuevas posibilidades para la aplicación de tecnologías de visión por computadora en otros aspectos de la agricultura.

