\section{Motivación}

La necesidad de mejorar la clasificación de semillas, es un proceso fundamental para aumentar la productividad agrícola, es la principal motivación de este proyecto. El uso de tecnologías emergentes como el aprendizaje automático ha demostrado ser efectivo para optimizar los costos y aumentar la producción, lo que impulsa el uso de estas herramientas en la agricultura. Se busca crear un sistema accesible para los pequeños agricultores que les permita automatizar tareas que tradicionalmente requerían un gran esfuerzo manual. Se utilizará un dispositivo compacto y de bajo costo como la ESP32-CAM.

La promoción de una agricultura más sostenible es otra motivación importante. Al mejorar la clasificación de semillas, se logra un uso más eficiente de recursos vitales como el agua, la tierra y los fertilizantes, lo que contribuye a una producción más responsable con el medio ambiente. Además, este proyecto podría abrir la puerta a nuevas aplicaciones de la visión por computadora en otros aspectos de la agricultura, como la recolección o el monitoreo de plagas, extendiendo los beneficios tecnológicos a diferentes áreas del campo.

Finalmente, este proyecto se alinea con la agenda 2030 de los Objetivos de Desarrollo Sostenible, particularmente con el ODS 2 (Hambre Cero) y el ODS 12 (Producción y Consumo Responsables). La promoción de una agricultura más sostenible es otra motivación importante. Al mejorar la clasificación de semillas, se logra un uso más eficiente de recursos vitales como el agua, la tierra y los fertilizantes, lo que contribuye a una producción más responsable con el medio ambiente. Además, este proyecto podría abrir la puerta a nuevas aplicaciones de la visión por computadora en otros aspectos de la agricultura, como la recolección o el monitoreo de plagas, extendiendo los beneficios tecnológicos a diferentes áreas del campo.
