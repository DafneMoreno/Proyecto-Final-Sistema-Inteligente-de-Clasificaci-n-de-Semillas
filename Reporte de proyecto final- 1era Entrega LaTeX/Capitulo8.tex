\section{Alcances y limitaciones}

El presente proyecto se compromete a desarrollar un prototipo funcional para la clasificación automatizada de semillas utilizando una cámara ESP32-CAM, con el objetivo de mejorar la eficiencia en la selección de semillas. El alcance del trabajo incluye la implementación de un sistema capaz de capturar imágenes de las semillas, procesarlas mediante algoritmos de visión por computadora y clasificarlas de acuerdo con parámetros predefinidos como su tamaño, color y forma. El sistema será probado en condiciones controladas de laboratorio para garantizar su funcionamiento dentro de un entorno predecible y adecuado. Además, se espera que esta solución sea accesible económicamente, permitiendo que pequeños agricultores puedan beneficiarse de una tecnología que tradicionalmente ha sido de alto costo.

Sin embargo, el proyecto presenta varias limitaciones que deben ser consideradas. La primera de ellas está relacionada con las condiciones de iluminación. Dado que el sistema depende de la captura de imágenes para su funcionamiento, cualquier variación en la iluminación podría afectar negativamente la calidad de las imágenes obtenidas y, por lo tanto, la precisión de la clasificación. El control de la iluminación es limitado a las condiciones de prueba en un entorno específico, y el rendimiento del sistema en ambientes no controlados podría no ser tan fiable.

Otra limitación importante es la capacidad del hardware. La ESP32-CAM es un dispositivo compacto y económico, pero sus recursos de procesamiento y memoria son limitados. Esto restringe el tipo y la complejidad de los algoritmos de aprendizaje automático y visión por computadora que se pueden implementar. Aunque se buscará optimizar el uso de recursos, las limitaciones inherentes del hardware podrían afectar la velocidad y precisión del sistema, especialmente al manejar grandes volúmenes de datos o imágenes de alta resolución.

Además, la variabilidad en las características físicas de las semillas representa otro reto. Las semillas pueden variar considerablemente en términos de tamaño, color y textura, incluso dentro de la misma especie. Este nivel de variabilidad puede dificultar que el sistema logre una clasificación precisa sin ajustes o recalibraciones adicionales, lo que podría limitar su aplicación en ciertos tipos de semillas o en condiciones más extremas.

Finalmente, el proyecto también está condicionado por los recursos materiales y de tiempo disponibles. El uso de un dispositivo de bajo costo implica que no se podrá contar con tecnologías adicionales que podrían mejorar la precisión del sistema, como sensores complementarios o algoritmos más avanzados. Además, las pruebas estarán restringidas a un entorno y periodo específico, lo que limita la capacidad de evaluar el sistema en diferentes contextos geográficos o ambientales. Estos factores podrían restringir el alcance del proyecto y generar la necesidad de futuras investigaciones para adaptar y mejorar el sistema a condiciones más diversas.