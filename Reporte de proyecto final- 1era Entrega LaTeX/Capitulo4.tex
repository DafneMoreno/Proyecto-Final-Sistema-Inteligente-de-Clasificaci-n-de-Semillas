\section{Antecedentes teóricos}

En las últimas décadas, la aplicación de tecnologías de visión por computadora para la clasificación y detección de semillas ha experimentado un desarrollo significativo. Diversos investigadores han abordado este desafío utilizando diferentes enfoques y tecnologías. La implementación de sistemas automatizados para la clasificación de semillas representa un avance importante en la agricultura de precisión, mejorando la eficiencia y precisión en los procesos de selección. Uno de los trabajos pioneros en este campo fue realizado por Kumar y Singh \cite{qiu2018variety}, quienes desarrollaron un sistema de clasificación de semillas utilizando técnicas de procesamiento de imágenes y redes neuronales convolucionales. Su investigación demostró una precisión del 95\% en la clasificación de diferentes tipos de semillas, estableciendo una base sólida para futuros desarrollos en esta área. En el contexto de sistemas embebidos, Patil \cite{patil2020realtime} implementó un sistema de bajo costo utilizando una ESP32-CAM para la clasificación de frutas en tiempo real. Su trabajo demostró la viabilidad de utilizar dispositivos de recursos limitados para tareas de visión por computadora, alcanzando una precisión del 89\% en condiciones controladas de iluminación. La clasificación automatizada de semillas ha sido abordada desde diferentes enfoques. Por ejemplo, Zhang et al. (2021) utilizaron técnicas de aprendizaje profundo para clasificar semillas de arroz, logrando una precisión del 98\%. Su sistema empleaba características como forma, tamaño y textura para la clasificación, aunque requería hardware más potente que una ESP32-CAM \cite{jin2022identification}. 

\subsubsection{Antecedentes históricos de la clasificación de semillas}
A lo largo de la historia, la clasificación de semillas ha sido un proceso crucial en la agricultura, ya que influye directamente en la calidad de los cultivos y, por ende, en el rendimiento de las cosechas. Originalmente, este trabajo se realizaba de manera completamente manual. Los agricultores, mediante una inspección visual, eran los encargados de separar las semillas basándose en características visibles como el tamaño, el color o la forma. Sin embargo, este método, además de ser laborioso y tedioso, estaba sujeto a errores debido a la variabilidad en la percepción humana. Lo que para una persona podía parecer una semilla adecuada, para otra podría no serlo, y esa falta de uniformidad en la selección resultaba en una calidad inconsistente en las cosechas.

Con el tiempo y el avance de la mecanización agrícola a mediados del siglo XX, comenzaron a implementar herramientas y dispositivos que facilitan esta labor. Uno de los primeros avances fue el uso de tamices, herramientas parecidas a coladores grandes que permitían separar las semillas según su tamaño. También se introdujeron separadores por peso, los cuales identifican y clasifican las semillas dependiendo de su densidad. Estos dispositivos representan una mejora significativa en la eficiencia del proceso, pero aún presentaban limitaciones. Por ejemplo, no podían diferenciar con precisión entre semillas que, aunque tenían el mismo tamaño o peso, variaba en características más sutiles, como la forma o el color, aspectos que también influyen en la calidad de la semilla.

El verdadero cambio en la clasificación de semillas ocurrió con el desarrollo de tecnologías más avanzadas en las décadas posteriores. En los años 80, la llegada de la visión por computadora marcó un punto de inflexión. Esta tecnología permitió que las máquinas "vieran" las semillas de una manera mucho más precisa, utilizando cámaras y sistemas computacionales capaces de analizar imágenes y detectar diferencias que serían imperceptibles para el ojo humano. Si bien en sus inicios estos sistemas son costosos y difíciles de implementar, sentaron las bases para las futuras innovaciones en el campo de la agricultura.

Con el tiempo, la integración de la inteligencia artificial (IA) y el aprendizaje automático (machine learning) ha mejorado aún más los sistemas de clasificación. A través de estos avances, las máquinas no solo pueden ver las semillas, sino que también pueden aprender de sus características y realizar clasificaciones más rápidas y precisas. Estos sistemas automatizados son capaces de analizar en tiempo real atributos como el tamaño, la forma y el color, permitiendo una inspección detallada que sería imposible de realizar de forma manual con la misma velocidad y precisión.

Hoy en día, la visión por computadora y la IA se han combinado para ofrecer soluciones que permiten una mayor precisión en la clasificación de semillas, lo que reduce significativamente los errores humanos y aumenta la eficiencia en la agricultura. La integración de estas tecnologías ha mejorado enormemente la capacidad de los agricultores para seleccionar las mejores semillas y, por lo tanto, obtener mejores cosechas.

\subsubsection{El uso de dispositivos embebidos y la democratización de la tecnología}
Más recientemente, el uso de dispositivos embebidos como la ESP32-CAM ha permitido llevar estas innovaciones tecnológicas a un nivel más accesible para los agricultores. La ESP32-CAM, un pequeño y económico dispositivo que incluye una cámara y conectividad Wi-Fi, ha demostrado ser una herramienta valiosa para la clasificación de semillas en tiempo real. A diferencia de los sistemas de visión por computadora más antiguos y costosos, la ESP32-CAM permite realizar análisis de imágenes con una inversión mucho menor, lo que ha democratizado el acceso a esta tecnología.

Este dispositivo, cuando se combina con bibliotecas de software como OpenCV y TensorFlow Lite, es capaz de ejecutar algoritmos de procesamiento de imágenes y aprendizaje automático, incluso en entornos de recursos limitados. Gracias a estos avances, la ESP32-CAM puede tomar fotos de las semillas, procesarlas y clasificarlas casi de manera instantánea, permitiendo que los agricultores trabajen con una mayor eficiencia sin necesidad de equipos complejos y costosos.

El impacto de esta tecnología en la agricultura moderna es significativo. Permite a los agricultores no solo ahorrar tiempo, sino también optimizar el uso de recursos, lo que se traduce en una mayor sostenibilidad. Además, al reducir la dependencia de la mano de obra para tareas repetitivas y tediosas, los agricultores pueden centrar sus esfuerzos en otras áreas importantes de la producción.

\subsubsection{Proyectos similares}
En los últimos años, se han desarrollado diversos sistemas de visión artificial utilizando dispositivos embebidos como la ESP32-CAM para tareas de clasificación y conteo. A continuación, se presentan algunos trabajos relevantes en este campo:

\begin{itemize}
    \item \textbf{Clasificación de Frutas:} Kim \cite{kim2022embedded} desarrollaron un sistema de clasificación de frutas utilizando una ESP32-CAM y redes neuronales convoluciones ligeras. Su sistema logró una precisión del 94.3\% en la clasificación de 5 tipos diferentes de frutas, demostrando la viabilidad de implementar modelos de deep learning en dispositivos con recursos limitados.
    \item \textbf{Conteo Automatizado de Productos:} Martínez-López \cite{martinez2023automated} implementaron un sistema de conteo automatizado para una línea de producción utilizando una ESP32-CAM. El sistema procesaba las imágenes en tiempo real y enviaba los datos a una base de datos en la nube, alcanzando una precisión del 98\% en el conteo de objetos en movimiento.

    \item \textbf{Clasificación de Granos:} Un trabajo particularmente relevante es el de Zhang \cite{zhang2021rice}, quienes desarrollaron un sistema de clasificación de granos de arroz utilizando una ESP32-CAM. Su enfoque combinó técnicas de procesamiento de imágenes tradicionales con un modelo de machine learning ligero, logrando una precisión del 91\% en la clasificación de diferentes variedades de arroz.
     
    \item \textbf{Control de Calidad en Agricultura:} Patel y Singh \cite{patel2023quality} implementaron un sistema de control de calidad para la clasificación de tomates utilizando una ESP32-CAM. Su sistema podía detectar defectos superficiales y clasificar los tomates según su madurez con una precisión del 89\%.
\end{itemize}