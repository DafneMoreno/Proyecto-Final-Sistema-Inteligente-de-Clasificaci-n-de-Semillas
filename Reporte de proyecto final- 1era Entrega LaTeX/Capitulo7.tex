\section{Justificación}

En la agricultura moderna, la clasificación precisa de semillas es una necesidad de alta precisión, ya que tiene un impacto directo en la calidad de los cultivos y en la eficiencia de la producción agrícola. Actualmente, la mayoría de los métodos de clasificación de semillas son manuales o mecánicos, lo que los hace propensos a errores humanos y con baja eficiencia en la diferenciación de características sutiles. La productividad agrícola y, en consecuencia, el ingreso de los agricultores, se ven afectados negativamente por esta situación. La FAO ha informado que, en algunos cultivos, la clasificación incorrecta puede generar pérdidas que alcanzan hasta el 10 \% de la producción total. Implementar el sistema automatizado propuesto en este proyecto, que utiliza visión por computadora, no solo mejoraría la precisión en la clasificación, sino que también reduciría costos al optimizar la utilización de los recursos en la producción.


El objetivo de este proyecto es automatizar el proceso de clasificación de semillas, ya que esto tiene un impacto directo en la productividad agrícola, la reducción de costos y la sostenibilidad ambiental. La combinación de tecnologías avanzadas como la visión por computadora y el aprendizaje automático en un dispositivo compacto y económico como ESP32-CAM ofrece una solución innovadora y accesible para los agricultores, especialmente en regiones donde el acceso a tecnología de alto costo es limitado. Además, al disminuir el trabajo manual, los agricultores obtienen una mejor calidad de vida y pueden concentrarse en otras actividades productivas. Los pequeños agricultores obtienen mayores beneficios al reducir los costos de clasificación de semillas, lo que fortalece las economías locales y mejora la seguridad alimentaria en sus comunidades.

