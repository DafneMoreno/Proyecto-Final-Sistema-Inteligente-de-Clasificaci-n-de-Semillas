\section{Descripción del problema}
La clasificación de semillas, uno de los pasos más críticos en la agricultura, afecta directamente a la calidad de la cosecha. Hasta ahora, la selección de semillas en función de características visibles como el tamaño, la forma, el color y la textura ha sido su naturaleza. La mayoría de las veces, también se realiza de forma manual. Sin embargo, la clasificación manual es muy imprecisa, hace perder tiempo y es costosa. Debido a las limitaciones de la precisión en el tiempo y el costo de la clasificación manual, los agricultores experimentan enormes dificultades debido a la disparidad física de las semillas, la fatiga visual y la subjetividad de la percepción humana. Esta clasificación no uniforme da lugar a enormes errores, por lo tanto, en la calidad de la cosecha producida y su producción agrícola final.Los errores de clasificación pueden provocar una menor germinación, menores rendimientos de los cultivos, mayor vulnerabilidad a las enfermedades y, en general, una calidad inferior en los productos agrícolas.
\\
Además, la población mundial sigue aumentando, lo que aumenta la demanda mundial de alimentos y la necesidad de eficiencia y precisión en los métodos de producción. La necesidad de una clasificación precisa de las semillas maximiza el rendimiento de los cultivos y minimiza el desperdicio. Si bien ha habido avances tecnológicos en la maquinaria agrícola en los últimos tiempos, existen varios países en desarrollo donde los agricultores aún realizan el proceso de forma manual o utilizando equipos muy simples diseñados para ese fin. Esto se debe a que los equipos de clasificación de semillas automatizados son muy caros y complicados, están diseñados principalmente para grandes explotaciones y están fuera de su alcance. Por esta razón, existen muchas ineficiencias en la producción agrícola con pérdidas económicas bastante grandes.
\\Uno de los desafíos clave al desarrollar un sistema de clasificación automatizado de semillas son las diferencias extremadamente altas en las características físicas, incluso dentro de la misma especie. Es decir, existen factores muy variados como el tamaño, la forma, y color, que tienen su influencia significativa. Su variabilidad se debe tanto a factores genéticos como ambientales. Además, las condiciones de captura fotográfica como la iluminación o el ángulo de la toma también son críticas. Por lo tanto, una característica esencial de un sistema eficaz es su capacidad de tomar en cuenta todos estos factores y proporcionar resultados consistentes en varias condiciones.
Por último,  está el costo de la implementación. Los agricultores de pequeños y medianos campos no pueden financiar la inversión en tecnologías avanzadas, lo que mantiene el uso de la clasificación manual. En esta situación, el sistema de clasificación de semillas basado en tecnología accesible y económica, como ESP32, tiene sentido. Dado que el dispositivo puede compartir la funcionalidad de la grabación de imágenes con un microcontrolador económico pero eficiente, ofrece una oportunidad para la implementación de la clasificación automatizada a un costo mucho menor al de los sistemas tradicionales.